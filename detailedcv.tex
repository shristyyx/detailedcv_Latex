\documentclass{resume} % Use the custom resume.cls style

\usepackage[left=0.4 in,top=0.4in,right=0.4 in,bottom=0.4in]{geometry} % Document margins
\newcommand{\tab}[1]{\hspace{.2667\textwidth}\rlap{#1}} 
\newcommand{\itab}[1]{\hspace{0em}\rlap{#1}}
\name{Shristy Thakur} % Your name
% You can merge both of these into a single line, if you do not have a website.
\address{+91 9368520229 / Pune, India} 
\address{Contact : \href{mailto:contact@shristythakur2003@gmail.com}{ Gmail /}  \href{https://github.com/shristyyx}{Github /}
\href{https://codeforces.com/profile/zoyaaax}{Codeforces /}  \href{https://www.linkedin.com/in/shristy-thakur-70a048226/}{Linkedin}}  %

\begin{document}


%CAREER OBJECTIVE
\begin{rSection}{Career Objective}

{
I am looking forward to working on Machine Learning and Data Analysis projects, along with learning new technologies and knowing more people, and learning from their experiences. \\
\\}
\end{rSection} 



%EDUCATION SECTION
\begin{rSection}{Education}

{\bf Bachelor of Engineering}, Army Institute of Technology, Pune\hfill {2021-Now}\\
Branch: Electronics and Telecommunication Engineering.

{\bf Secondary School}, Delhi Public School Ghaziabad, Dehradun  \hfill {2020-2021}\\
Percentage: 95.2 / Salutatorian Award / Head Girl.

{\bf High School}, Shemford Doon, Dehradun  \hfill {2018-2019}\\
Percentage: 97.6 / Valedictorian Award / House Captain.

\end{rSection}




\begin{rSection}{PROJECTS}

\vspace{-1.25em}
\item \textbf{ML Prediction models} {| (Python) |  Built two machine learning models which uses a Kaggle datasets- one  to predict house prices based on various input features, and another that predicts the probability of a person having diabetes after taking in user input, deployed in web page using StreamLit. \href{https://github.com/shristyyx/House-Prize-preditiction}{(HousePrize)} and \href{(https://github.com/shristyyx/DiabetesTestAImodel)}{(Diabetes)}} 


\item \textbf{Python Games and Interfaces} {| (Python) | Built two car racing games using Pygame and two projects using Tkinterface, A Countdown Calendar and a Zodiac window that displays the user's zodiac traits. \href{https://github.com/shristyyx?tab=repositories}{(PythonProjects)}}



\item \textbf{Spotify Clone} {| (HTML, CSS) | Built a spotify clone with embedded youtube videos. \href{https://github.com/shristyyx/Spotify-Clone}{(Youtify)} }


\end{rSection} 



%SKILLS SECTION
\begin{rSection}{SKILLS}

\begin{tabular}{ @{} >{\bfseries}l @{\hspace{6ex}} l }
Technical Skills 
& Machine Learning, Data Analysis, Object Oriented Programming. 

\\&  Data Structures and Algorithms, Competitive Programming,
\\&Data Base Management System.\\
Programming Languages & C++, Python, Java, C, HTML, CSS, SQL.\\
Frameworks & MySQL, NumPy, Pandas, Matplotlib, Seaborn, Linux.\\
Developing Tools & GitHub, VS Code, Pycharm, Jupyter Notebook, Arduino, AutoCad, Figma.\\

\end{tabular}\\
\end{rSection}





%CONFERENCES ATTENDED
\begin{rSection}{Conferences attended }

\textbf{Impactful LinkedIn Building by UST  } \hfill 10.03.23\\
Attended a session conducted by UST on building an impactful LinkedIn profie, followed by games to make the event more fun.


\textbf{Idea Presentaion Session by American Express } \hfill 7.05.23\\
Attended an online session by American express for Make-A-Thon(a hackathon conducted for women techmakers) on a good idea pitching and presentaion. \\


\end{rSection} 




%LEADERSHIP EXPERIENCES
\begin{rSection}{Leadership Experience}

\textbf{Joint Secretary } \hfill June 2022- now\\
Magazine Board (Graphic Designer) and Information Security and Digital Forensics Club  \hfill \textit{\\} {
1. Managed Alumni Meet 2022 with the assistance of 30 first-year members.\\
2. Conducted biggest slam poetry event in Pune (team size-30) called Reverse in which 20 poets performed. 
}


\textbf{Member} \hfill Nov 2021- June 2022\\
Entrepreneurship Cell(Ecell), Debate Dramatics and Quiz club\\\hfill \textit{}
\\
\\
\end{rSection} 



%ACHIEVEMENTS SECTION
\begin{rSection}{TECHNICAL ACHIEVEMENTS}
\vspace{-1.25em}

\item \textbf{Hackerrank Badges} {earned three golden badges (five stars) on Hackerank in Problem Solving, Python, and Java.
\href{https://www.hackerrank.com/zoyaaax}{(HackerRank)}   }


\item \textbf{Codeforces best rank} {secured a global rank of 3590 in Codeforces Round 839 (Div. 3) and maximum contest rating of 1184. 
\href{https://codeforces.com/profile/zoyaaax}{(Codeforces)}   

\item \textbf{Codechef best rank} {secured a global rank of 1050 in Codechef Starters 58 (Div. 4).
\href{https://www.codechef.com/users/shristythakur}{(CodeChef)}   }
\\
}



\end{rSection} 


 %PERSONAL SUMMARY
\begin{rSection}{Personal Summary }

Being a second year student, I am particularly enthusiastic about machine learning and data science, as well as web designing and web development.
I have gained hands-on experience with several programming languages such as Python, C++, Java, and C. However, I find myself most comfortable using C++ and Python. When it comes to problem-solving with data structures and algorithms, I prefer using C++. I frequently participate in coding contests on platforms like Codeforces and Codechef. In addition, I enjoy using Python to develop projects in machine learning, which has resulted in the creation of numerous small projects.\\


\end{rSection} 



%HOBBIES AND INTERESTS
\begin{rSection}{Hobbies and Interests}

{
I like spending my free time playing sports, decorating photo journals and writing poems. 
 My interests include Problem Solving in DSA, learning new programming languages and algorithms and participating in coding contests. I also enjoy learning new technologies. \\

}
\end{rSection} 


\end{document}
